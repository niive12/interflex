\documentclass[11pt,a4paper]{article}
\usepackage[utf8]{inputenc}
\usepackage[english]{babel}
\usepackage{amsmath}
\usepackage{amsfonts}
\usepackage{amssymb}
\usepackage[dvips]{graphicx}
\usepackage{float}
\usepackage{picture}
\DeclareGraphicsExtensions{.bmp,.png,.pdf,.jpg}
\usepackage[left=2cm,right=2cm,top=2cm,bottom=2cm]{geometry}
\author{Xabier Martinez}
\title{Guide/Summary EiT}
\date{14/09/2014}
\begin{document}
\maketitle
\begin{abstract}
\begin{center}
This is a guide/summary I've done with all the information of EiT.
\end{center}
\end{abstract}
\section{Collaboration}
- \emph{Team Role} $\to$ A tendency to behave, contribute and interrelate with others.\\
- \emph{Function Role}\\ \\
\textbf{Team Role} $\to$ Behaviour (Intellect, Personality, Role Learning, Enviroment, Experience, Values and Motivation)\\
In a team each one has a role:

	- \emph{Action Roles} $\to$ Shapers, Implementers, Completer finisher
	
	- \emph{Social Roles} $\to$ Co-ordinator, Teamworker, Investigator
	
	- \emph{Thinking Roles} $\to$ Specialist, Monitor Evaluator, Plant\\ \\

- \underline{Plant} $\to$ Highly creative and good solving problems.

- \underline{Monitor evaluator} $\to$ Provide a logical eye, make impartial judgements.

- \underline{Co-ordinator} $\to$ Focus on team's objectives and delegate work appropiately.

- \underline{Resource investigators} $\to$ Privide inside knowledge on the opposition.

- \underline{Implementers} $\to$ Plan a practical, workable strategy and carry it out.

- \underline{Completer} finisher $\to$ Used at the end of a task.

- \underline{Teamworkers} $\to$ Help the team to gel.

- \underline{Shapers} $\to$ Ensure that the team kept moving and did not loose focus or momentum.

- \underline{Specialist} $\to$ Individual with in-depth knowledge of a key area.\\ \\

1 - \textbf{Group Dynamics} $\to$ Stability $\to$ Balanced individual $\to$ They need pressure

\hspace{3.9cm} $\to$ Restlesness $\to$ Easy to influence $\to$ Try to link with high expectations

\subsection{Stages of a Project}
- Identify needs and describe goals

- Ideas and solution suggestions

- Plans+strategy

- Contact to stakeholders

- Organizing tasks and responsabilities

- Implementation and operation

\subsection{Team SWOT}
\begin{figure}[hbtp]
\begin{center}
\includegraphics[scale=0.5, natwidth=830, natheight=499]{swot.eps}
\end{center}
\end{figure}
By identifying our Team Roles, we can ensure that we use our strengths to advantage and that we manage our weakness as best we can. \\

Each member could have more than one role in the team. \\
\hspace{10cm} - Preferred team role.\\
\hspace{10cm} - Manageable roles. \\
\hspace{10cm} - Last preferred roles. \\

Allowable weakness $\to$ It's a weakness, which we use in a certain contex, so it's because a strength.\\

Things-To-Do-And-Not-To-Do for each role $\to$ Read BELBIN-UK-2012-ThingsToDoAndNotToDo\\

\section{Business}
1 - \textbf{Understanding and innovating our Business Model is crucial.} Competitive advantage is achieved through focused and innovative business models.\\

2 - \textbf{What actually is a Business Model?}

The Business Model of a company is a simplified representation of its business logic. It describes what a company offers its customers, how it reaches them and related to them, through which resources, activities and partners it achieves this and finally, how it earns money. The business model is usually distinguished from the business process model and the organization model.\\

3 - \textbf{The Building block of a Business Model.}
\begin{figure}[hbtp]
\begin{center}
\includegraphics[scale=0.5]{Fotos/business1.eps}
\end{center}
\end{figure}

* Customer segments $\to$ Our group of customers with distinct characteristics.

* Value proposition $\to$ The bundles of products and services that satisfied of customer segments needs.

* Distribution channels $\to$ The channels through which we communicate with our customers and through we offer our value proposition.

* Customer relationship $\to$ The types of relationships we entertain with each customer segments.

* Reverse streams $\to$ The streams through which we earn our revenues from our customers.

* Key resources $\to$ The key resources on which our business model is built.

* Key activities

* Partner network $\to$ The partners and suppliers we work with.

* Cost structure $\to$ The cost we incur to run our business model.\\

4 - \textbf{The process of Describing, Assesing and Improving a Business Model.}

\emph{Business Model} $\to$ Draw a clear picture of our existing business planning.

\hspace{2.7cm} $\to$ Assessment of the strength, weakness, opportunities and threats of our current business model.

\subsection{Describing a Business Model step by step}
(It should be a team effort)

To sketch out a company's business model the team will describe each of the 9 business model building blocks and highlight the linkages between them.\\

* \emph{\textbf{Customer Segments}} $\to$ Who are our customers?

\emph{Objective} $\to$ Regroup them in terms of different needs and ways of reaching them.\\
\begin{figure}[hbtp]
\begin{center}
\includegraphics[scale=0.5]{Fotos/business2.eps}
\includegraphics[scale=0.5]{Fotos/business22.eps}
\end{center}
\end{figure}
After answering the questions, we should describe them a little bit more (demographic and geographical information, core needs, aspirations, ...).

Finally, we should add some statistical information (number of current customers, profitability, growth potential, etc.).\\

* \emph{\textbf{Value Proposition}} $\to$ What do we offer each of our client segments?

\emph{Objective} Identify the value you create for each distinct customer segment by describing the bundle of products and services you offer them.\\
\begin{figure}[hbtp]
\begin{center}
\includegraphics[scale=0.5]{Fotos/business3.eps}
\end{center}
\end{figure}

* \emph{\textbf{Channels}} $\to$ How do we reach of our client segment?

\emph{Objective} Identify the channels through which we offer our value propositions to each customer segments. (Advertising, retail outlets, sales teams, websites, conferences, etc.)\\
\begin{figure}[hbtp]
\begin{center}
\includegraphics[scale=0.5]{Fotos/business4.eps}
\end{center}
\end{figure}

* \emph{\textbf{Customer Relationship}} $\to$ How do we relate to our clients over time?

\emph{Objective} $\to$ Identify which types of relationships you have built in which you maintain with each customer segment.\\
\begin{figure}[hbtp]
\begin{center}
\includegraphics[scale=0.5]{Fotos/business5.eps}
\end{center}
\end{figure}

* \emph{\textbf{Revenue Streams}} $\to$ How do we earn money?

\emph{Objective} $\to$ Identify which types of revenue streams we earn from each of our customers segments and value prepositions.

Revenue streams come in the form of selling, lending, licensing, commissions, transaction fees or advertising fees.
\begin{figure}[hbtp]
\begin{center}
\includegraphics[scale=0.5]{Fotos/business6.eps}
\end{center}
\end{figure}

This will help is draw a clearer picture on each segment's contribution to overall revenues.\\

* \emph{\textbf{Key Resources}} $\to$ Based on which assets are we running our business?

\emph{Objective} $\to$ Identify the key tangible and intangible resources which are the fundamental of our business model.\\
\begin{figure}[hbtp]
\begin{center}
\includegraphics[scale=0.5]{Fotos/business7.eps}
\end{center}
\end{figure}

* \emph{\textbf{Key Activities}} $\to$ What key activities do we need to run our business model?\\
\begin{figure}[hbtp]
\begin{center}
\includegraphics[scale=0.5]{Fotos/business8.eps}
\end{center}
\end{figure}

* \emph{\textbf{Partner Network}} $\to$ With which partners do we leverage our business?

\emph{Objective} $\to$ Outline with which partners and suppliers we work to implement our business model.\\
\begin{figure}[!H]
\begin{center}
\includegraphics[scale=0.5]{Fotos/business9.eps}
\end{center}
\end{figure}

* \emph{\textbf{Cost structure}} $\to$ Where are our most important costs?

\emph{Objective} $\to$ Identify our most important cost position resulting from our business model.\\
\begin{figure}[!H]
\begin{center}
\includegraphics[scale=0.5]{Fotos/business10.eps}
\includegraphics[scale=0.5]{Fotos/business1010.eps}
\end{center}
\caption{text underneath}
\end{figure}

Cost structure is a direct result of all the other building blocks.\\

5 - \textbf{Business Model Assessment}\\

6 - \textbf{Business Model Innovation and Improvement}

After having drawn a clear picture of our current business model, we can now build on the conclusion of our business model assessment including strengths, weakness, opportunities and threats.

\section{Starting Up How To Write A Business Plan}
- \textbf{Round 1}: Concept and presentation of a business idea. What problem the idea solves, what is new about their product, why customers would want to use it, who the target group is an who is going to pay for the product.\\

-\textbf{Round 2}: Assessing the feasibility and potential of the start-up company.

Are you able and allowed to produce your product on the necessary scale? In what way is the product better than its competition? Who are the competitors? What is the current and long term market potential? What price are your customers willing to pay for your products, and will that be enough to make profit? The analyses of this will eventually end up in a business plan.\\

\textbf{Round 3}: Preparation and presentation of the business plan.

It must report the product idea, the profiler and competencies of the management team, the marketing possibilities of your product, the way your company will operate, the detailed time planning of the realization of your company, the risk involved and the financial planning.\\

- \emph{Content of a convincing business idea}\\
$\to$ what is the customer benefit; or, what problem does the idea solve?\\

$\to$ What is the market?\\

$\to$ How will it make money?\\

$\to$ Customer service\\

$\to$ Market\\

$\to$ What is the market for the product or service offered?\\

$\to$ How large is it?\\

$\to$ What are the primary target groups or segments?\\

$\to$ Revenue mechanism\\ \\

- \emph{Presenting the business idea}\\

- \emph{From management team to 'Dream Team'}

This will not only enable you to make best use of the abilities of those involve, but will also reveal any gap.

\section{Marketing}
$\to$ How to analyse your market and the competitions.

$\to$ How to choose your target market.

$\to$ How to determine your marketing strategy.

\subsection{Market and competition}
- Market size in terms of number of customers, the number of units and the total sales in Euro.\\

- Karget market $\to$ Who the most important suppliers in the market are, what their market share is, how they operate, and what their strength and weakness are. Name your competitors specifically, and describe why and how your company will be better.\\

- Choosing the t\textbf{arget market} $\to$ Who exactly are your customers?\\
\hspace{3cm} \emph{Customer Segmentation}$\to$ Apply appropriate criteria to group your potential customers. Criteria are appropriate if they produce customer groups that are internally consistent and it must also be applicable to product design, pricing, publicity and distribution. Two purpouses:
\begin{itemize}
\item It helps define the market that your product can reach.\\
\hspace{3cm} For industry good (Demographic, Operational, Purchasing Behaviour, Situational Factors)
\item It helps you design a specific - and thus more effective - marketing strategy for each customer segment.\\ \\
\end{itemize}

- Choosing the \textbf{Target Segment}

% Starting up how to write a Business Plan (page 82.)
% http://www.youtube.com/watch?v=F3LM23STOic&feature=youtu.be
% http://idea-bmc.dk
% NTNU_CreativityLab (page 4.)
\end{document}
