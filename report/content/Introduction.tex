\section{Introduction}

This report contains a business plan for Interflex, that should legitimize how the company is going to make profit and why future stakeholders should invest in Interflex. 
We have documented how the team has collaborated as this project is part of the course Experts in Team.
This project has addressed issues from the analysis of markets and customers, to the different techniques for developing and improving teamwork, while having the main product to sell in focus.

Robotics have changed the industrial area, due to the effectiveness of automation. But the complexity of automation and robotics are causing costs to the end users in form of time and money. Every time a specific robot needs to solve a new task, a new program has to be made for this robot. This is not a flexible way of handling tasks. 

Our project is narrowed down to the area of welding within an automated industry. Within this area we want to establish a product that provides flexibility to welding robots, so they ease the step of reprogramming when a new task occurs. Programming robots is a complex process, that requires technical skill. This is not a cheap process either, so the end users would like to avoid the process of reprogramming, also due to consumption of time. If this process can be avoided or made easier, the end customers can save a good amount of money. 

We provide a solution to the problem mentioned above, which does not require programming. The product itself is an add-on for welding robots, that allows the robot to detect welding seams with minimal user interaction. The add-on will provide coordinates, used to create a welding-path for the robot. 
The product consists of a micro computer, which is used to make calculations based on input from a camera and a laser range scanner. 

What is sought with this project is to create a business plan, understanding the process of starting a business, while learning new techniques of collaboration and innovation, and ending up drawing conclusions from the results.
