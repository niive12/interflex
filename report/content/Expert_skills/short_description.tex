
The product we want to create is an intelligent sensor.
This sensor will allow for a new kind of programming that aims to be more intuitive and faster than current methods.
We propose a tool which can be attached to and interface with an existing welding robot. 
By using a special pen to mark the places an object should be welded this sensor can follow the markings to program the robot automatically.
We want to combine the use of computer vision and lasers to determine the localization of the object and the welds.
We want to let the workers change settings via a touch interface so the robot knows how to handle the markings. 

The detection requires a two part localization. 
From a perspective position or sets of positions the sensor will be able to see all markings.
The robot should be able to determine where the markings is on the object. 
This is what we refer to as global localization.
To do this we use a camera and computer vision to locate and move the robot closer to the markings.
The sensor will pick the first marking it can find and move closer towards that. 
When a marking is found we need to locate the gap between two objects. 
When a gap exist within a marked area that is treated as a place that should be welded.
By using lasers we can determine where there is a gap as well as determine the angles between the objects.
We call this part local localization.
When a weld is completed the robot will return to the perspective position to find the next marking until there is no more markings left.