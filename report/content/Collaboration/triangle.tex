\subsection{Competence triangle}
\tikzstyle{theo} = [font=\footnotesize, rotate=60,  draw, rectangle, fill=blue!10, text=black, text width = 3.7em, minimum height =4em, text centered, node distance=1.7cm]
\tikzstyle{exp} =  [font=\footnotesize, rotate=-60, draw, rectangle, fill=blue!10, text=black, text width = 3.7em, minimum height =4em, text centered, node distance=1.7cm]
\tikzstyle{pers} = [font=\footnotesize,             draw, rectangle, fill=blue!10, text=black, text width = 3.7em, minimum height =4em, text centered, node distance=1.7cm]

\begin{figure}[!ht]
\centering
\begin{tikzpicture}
\node[draw, very thick, regular polygon,regular polygon sides = 3, minimum width= 7cm] (strengths) at (0,0) {};

\node[rotate=60, yshift=-0.5cm, xshift=-0.3cm] (theoretical) at (strengths.135) {Theoretical knowledge};
\node[rotate=-60, yshift=-0.5cm, xshift=0.3cm] (experience) at (strengths.40) {Work skills and Experiences};
\node[yshift=0.5cm] (personal) at (strengths.270) {Personal strengths};

\node[theo, above of=theoretical, yshift = -0.2cm, xshift=-3cm] (ta) {Software development (4)} ;
\node[theo, right of=ta] (tb)                                        {Electrical circuits} ;
\node[theo, right of=tb] (tc)                                        {Databases} ;
\node[theo, right of=tc] (td)                                        {Maths (3)} ;
\node[theo, above of=ta, xshift=-2.5em, yshift=-0.2cm] (te)          {Control theory} ;
\node[theo, right of=te] (tf)                                        {Energy} ;
\node[theo, right of=tf] (tg)                                        {C++ programming (2)} ;
\node[theo, right of=tg] (th)                                        {C programming (3)} ;
\node[theo, right of=th] (ti)                                        {Java programming (1)} ;
\node[theo, above of=te, xshift=-2.5em, yshift=-0.2cm] (tj)          {Physics} ;
\node[theo, right of=tj] (tk)                                        {Mechanics} ;
\node[theo, right of=tk] (tl)                                        {RE tech} ;
\node[theo, right of=tl] (tm)                                        {Lasers} ;
% \node[theo, right of=tm] (tn)                                        {Waves} ;
\node[theo, right of=tm] (to)                                        {Process management} ;
\node[theo, right of=to] (tp)                                        {Computer simulation} ;

\node[exp, above of=experience, yshift = -0.2cm,xshift=-3cm] (ea) {Teaching non experts} ;
\node[exp, right of=ea] (eb)                                      {Auto\-mation} ;
\node[exp, right of=eb] (ec)                                      {Debating} ;
\node[exp, right of=ec] (ed)                                      {Wind turbines} ;
\node[exp, above of=ea, xshift=-2.5em, yshift=-0.2cm] (ee)        {Explo\-rative} ;
\node[exp, right of=ee] (ef)                                      {Program\-ming} ;
\node[exp, right of=ef] (eg)                                      {Planning} ;
\node[exp, right of=eg] (eh)                                      {Control} ;
\node[exp, right of=eh] (ei)                                      {Carpentry} ;
\node[exp, above of=ee, yshift=-0.2cm] (ej)                       {Network handling} ;
\node[exp, right of=ej] (ek)                                      {Software solutions} ;
\node[exp, right of=ek] (el)                                      {Self improvement} ;
\node[exp, right of=el] (em)                                      {Tool handling} ;
\node[exp, right of=em] (en)                                      {Effective} ;

\node[pers, below of=personal, yshift = 0.2cm, xshift = -7.3em] (pa) {Logical} ;
\node[pers, right of=pa] (pb)                                        {Patient (3)} ;
\node[pers, right of=pb] (pc)                                        {Compre\-hensive} ;
\node[pers, right of=pc] (pd)                                        {Objective} ;
\node[pers, below of=pa, xshift=-1.85em, yshift=0.1cm] (pe)          {Humorous} ;
\node[pers, right of=pe] (pf)                                        {Team player} ;
\node[pers, right of=pf] (pg)                                        {Calm in difficult situations} ;
\node[pers, right of=pg] (ph)                                        {Analytic} ;
\node[pers, right of=ph] (pi)                                        {Realistic} ;
\node[pers, below of=pe, xshift=-4.835em, yshift=0.1cm] (pj)         {Outgoing} ;
\node[pers, right of=pj] (pk)                                        {Observant} ;
\node[pers, right of=pk] (pl)                                        {Positive (2)} ;
\node[pers, right of=pl] (pm)                                        {Structured} ;
\node[pers, right of=pm] (pn)                                        {Effective} ;
\node[pers, right of=pn] (po)                                        {Communi\-cative} ;
\node[pers, right of=po] (pp)                                        {Willing to optimize solutions} ;



\end{tikzpicture}
\caption{Competence triangle \label{fig_triangle}}
\end{figure}

In order to learn more about each group member a competence triangle was created (figure \ref{fig_triangle}). 
The competence triangle separates competences that are on a personal, theoretical and experience level. 
This is done to get a better understanding of how people view themselves and what their education involves.
Each member wrote down 2-3 things about themselves and each item and its relation to the project was discussed.
The group has a lot of math and programming focused people, but lacks business oriented people.