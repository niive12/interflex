
\subsection{Competence Triangle}

\tikzstyle{theo} = [font=\footnotesize, rotate=60,  draw, rectangle, fill=blue!10, text=black, text width = 3.7em, minimum height =4.1em, text centered, node distance=1.7cm]
\tikzstyle{exp} =  [font=\footnotesize, rotate=-60, draw, rectangle, fill=blue!10, text=black, text width = 3.7em, minimum height =4.1em, text centered, node distance=1.7cm]
\tikzstyle{pers} = [font=\footnotesize,             draw, rectangle, fill=blue!10, text=black, text width = 3.7em, minimum height =4.1em, text centered, node distance=1.7cm]

\begin{figure}[!ht]
\centering
\begin{tikzpicture}[scale=0.7, every node/.style={scale=0.7}]
\node[draw, very thick, regular polygon,regular polygon sides = 3, minimum width= 7cm] (strengths) at (0,0) {};

\node[rotate=60, yshift=-0.5cm, xshift=-0.3cm] (theoretical) at (strengths.135) {Theoretical knowledge};
\node[rotate=-60, yshift=-0.5cm, xshift=0.3cm] (experience) at (strengths.40) {Work skills and Experiences};
\node[yshift=0.5cm] (personal) at (strengths.270) {Personal strengths};

\node[theo, above of=theoretical, yshift = -0.2cm, xshift=-3cm] (ta) {Software development (4)} ;
\node[theo, right of=ta] (tb)                                        {Electrical circuits} ;
\node[theo, right of=tb] (tc)                                        {Databases} ;
\node[theo, right of=tc] (td)                                        {Maths (3)} ;
\node[theo, above of=ta, xshift=-2.5em, yshift=-0.2cm] (te)          {Control theory} ;
\node[theo, right of=te] (tf)                                        {Energy} ;
\node[theo, right of=tf] (tg)                                        {C++ programming (2)} ;
\node[theo, right of=tg] (th)                                        {C programming (3)} ;
\node[theo, right of=th] (ti)                                        {Java programming (1)} ;
\node[theo, above of=te, xshift=-2.5em, yshift=-0.2cm] (tj)          {Physics} ;
\node[theo, right of=tj] (tk)                                        {Mechanics} ;
\node[theo, right of=tk] (tl)                                        {RE tech} ;
\node[theo, right of=tl] (tm)                                        {Lasers} ;
% \node[theo, right of=tm] (tn)                                        {Waves} ;
\node[theo, right of=tm] (to)                                        {Process management} ;
\node[theo, right of=to] (tp)                                        {Computer simulation} ;

\node[exp, above of=experience, yshift = -0.2cm,xshift=-3cm] (ea) {Teaching non experts} ;
\node[exp, right of=ea] (eb)                                      {Auto\-mation} ;
\node[exp, right of=eb] (ec)                                      {Debating} ;
\node[exp, right of=ec] (ed)                                      {Wind turbines} ;
\node[exp, above of=ea, xshift=-2.5em, yshift=-0.2cm] (ee)        {Explo\-rative} ;
\node[exp, right of=ee] (ef)                                      {Program\-ming} ;
\node[exp, right of=ef] (eg)                                      {Planning} ;
\node[exp, right of=eg] (eh)                                      {Control} ;
\node[exp, right of=eh] (ei)                                      {Carpentry} ;
\node[exp, above of=ee, yshift=-0.2cm] (ej)                       {Network handling} ;
\node[exp, right of=ej] (ek)                                      {Software solutions} ;
\node[exp, right of=ek] (el)                                      {Self improvement} ;
\node[exp, right of=el] (em)                                      {Tool handling} ;
\node[exp, right of=em] (en)                                      {Effective} ;

\node[pers, below of=personal, yshift = 0.2cm, xshift = -7.3em] (pa) {Logical} ;
\node[pers, right of=pa] (pb)                                        {Patient (3)} ;
\node[pers, right of=pb] (pc)                                        {Compre\-hensive} ;
\node[pers, right of=pc] (pd)                                        {Objective} ;
\node[pers, below of=pa, xshift=-1.85em, yshift=0.1cm] (pe)          {Humorous} ;
\node[pers, right of=pe] (pf)                                        {Team player} ;
\node[pers, right of=pf] (pg)                                        {Calm in difficult situations} ;
\node[pers, right of=pg] (ph)                                        {Analytic} ;
\node[pers, right of=ph] (pi)                                        {Realistic} ;
\node[pers, below of=pe, xshift=-4.835em, yshift=0.1cm] (pj)         {Outgoing} ;
\node[pers, right of=pj] (pk)                                        {Observant} ;
\node[pers, right of=pk] (pl)                                        {Positive (2)} ;
\node[pers, right of=pl] (pm)                                        {Structured} ;
\node[pers, right of=pm] (pn)                                        {Effective} ;
\node[pers, right of=pn] (po)                                        {Communi\-cative} ;
\node[pers, right of=po] (pp)                                        {Willing to optimize solutions} ;



\end{tikzpicture}
\caption{Competence triangle. The qualities are divided into three categories (Theoretical knowledge, Work Skills and Experiences, and Personal Strengths), and put on the side of the triangle that matches. If more than one person in the group has the quality, the number of people having this quality, is in parenthesis.} \label{fig_triangle}
\end{figure}

In order to learn more about each group member a competence triangle was created (figure \ref{fig_triangle}). 
The competence triangle separates competences that are on a personal, theoretical and experience level. 
This is done to get a better understanding of how people view themselves and what their education involves.
Each member wrote down 2-3 things about themselves and each item and its relation to the project was discussed.

The theoretical part is covered by many technical fields, for instance math, physics and programming. 
This signals a strong area of knowledge when it comes to calculating and developing solutions. 
This was used in the idea generation to what kind of product we could come up with when we combined our theoretical knowledge. The experience section covers communication by teaching and debating, but also experience from jobs. This can come in handy when you have to realize an idea. 


While everything listed in the competence triangle was things that we could provide, it also showed what we could not provide.
We did not have anyone with skills in making a business plan and we did not have anyone who knew much about idea generating, and the business part required in this project.

This made us aware, very early in the process that we would have problems in these areas that we would have difficulties solving. We used tools provided, like the picture brainstorm\todo[inline]{REFER TO SECTION} to generate ideas. The business plan we did by using a pregenerated model, which made it more manageable. Besides that, the early notice about the problems we would have, gave us more time to study. 

Overall, we did not use the competene trianle a lot directly, however, it was a good tool to get us talking and to learn something about eachother and the group. 
