\subsection{Belbin}

\begin{figure} [h!]
\includegraphics[width=\textwidth]{./graphics/Belbin_spiderweb}
\caption{Belbin Self-perception "Spiderweb"}
\label{belbinspider}
\end{figure}

\begin{table}[ht]
\centering
\begin{tabular}{|p{0.45\textwidth}|p{0.45\textwidth}|}
\hline
\multicolumn{1}{|c|}{\textit{Contribution:}}                                                      & \multicolumn{1}{c|}{\textit{Allowable Weaknesses:}}                          \\ \hline
\multicolumn{2}{|l|}{\textbf{Top 3 roles:}}                                                                                                                                      \\ \hline
\multicolumn{2}{|c|}{\textbf{Monitor Evaluator}}                                                                                                                                 \\ \hline
Sober, strategic and discerning. Sees all options and judges accurately.                          & Lacks drive and ability to inspire others. Can be overly critical to others. \\ \hline
\multicolumn{2}{|c|}{\textbf{Implementer}}                                                                                                                                       \\ \hline
Practical, reliable, efficient. Turns ideas into actions and organizes work that needs to be done.& Somewhat inflexible. Slow to respond to new possibilities.                   \\ \hline
\multicolumn{2}{|c|}{\textbf{Completer Finisher}}                                                                                                                                \\ \hline
Painstaking, conscientious, anxious. Searches out errors. Polishes and perfects.                  & Inclined to worry unduly. Reluctant to delegate.                             \\ \hline
\multicolumn{2}{|l|}{\textbf{Bottom 3 Roles:}}                                                                                                                                   \\ \hline
\multicolumn{2}{|c|}{\textbf{Shaper}}                                                                                                                                            \\ \hline
Challenging, dynamic, thrives on pressure. Has the drive and courage to overcome obstacles.       & Prone to provocation. Offends people's feelings.                             \\ \hline
\multicolumn{2}{|c|}{\textbf{Plant}}                                                                                                                                             \\ \hline
Creative, imaginative, free-thinking. Generates ideas and solves difficult problems.              & Ignores incidentals. Too preoccupied to communicate effectively.             \\ \hline
\multicolumn{2}{|c|}{\textbf{Resource Investigator}}                                                                                                                             \\ \hline
Outgoing, enthusiastic, communicative. Explores opportunities and develops contacts.              & Over-optimistic. Loses interest once initial enthusiasm has passed.          \\ \hline
\end{tabular}
\caption{Top/Bottom 3 Belbin Self-perception for the group}
\label{belbintable}
\end{table}

Figure \ref{belbinspider} is a team overview based on the results of the individual tests. 
It is fast to view which areas is not covered by any person when viewed in a spiderweb plot.
Table \ref{belbintable} shows the strong and weak roles for the team profiles.

It is very clear that the group has a major potential when it comes to developing solutions to perfection, while being able to investigate the different possibilities. 
This could be explained by the amount of specialists in the group.

It is also very clear that the group lacks drive and a key person to set the pace of the work processes. 
The groups has to be aware that the beginning of the project can potentially cause issues. 
This is due to the lack of Plants and Resource Investigators. 
The Plants provide creativity and innovation, while the Resource Investigators validates the possibility of an idea.

To compensate for this we use a strict planning and idea generation methods.

The weaknesses have been identified and it was time to take action to make our actions clear for the group. 
These allowable weaknesses as seen above is not always preventable. 
There has been an awareness about the lack of drive, from the beginning. 
This has led to long discussions about minor subjects, which required less attention than it got. 
For instance, the few shapers in the group has to be aware of their role. 

To keep people focused we divide people up in smaller groups so the big group discussions doesn't take all our time.

We have also identified the strong forces our group has to offer.
We try to turn any discussion in how to solve problems and how we can move forward. 

%This is due to the missing Plants who provide creativity and innovation to the group, together with the resource investigators who makes sure the actual ideas are possible at all.
 
%Since we already encountered difficulties in the beginning of the project, it is obvious that the group is missing ideas to work with. That could also explain why we decided to go with the basic idea that was described in the project introduction papers.

%The strengths and weaknesses can all be related to the group-SWOT test.

%<Insert comparison>.