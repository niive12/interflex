\subsection{Team SWOT Analysis} 
SWOT stands for: Strengths, Weaknesses, Opportunities, and Threats. As the name says, it is a tool, which helps seeing what opportunities and threats a team has, from its strengths and weaknesses. 
 
Table \ref{SWOT-table} shows the SWOT-analysis of the team. 
By looking at a SWOT analysis of each person we can determine what threats can be resolved by another group member.
If a trait is represented as both a strength and a weakness in the SWOT, it is removed. This way the SWOT will represent only the strengths and weaknesses that the group may have. This will help identify the threats the group will have to deal with.

We combined that with the strengths and weaknesses from the Belbin and compared that to the three top and bottom roles, so we extracted some ideas from there.

The strengths leads us to discover what possibilities the team has and the weakness leads us to discover the team's threats. 

As a team, we are very good at solving and working with problems, but at the same time we are having difficulties finding these problems. 
This means that the team should be aware of difficulties, especially in the beginning of the project.

We are not good at ending discussions. We have tried to stop them after a certain time, and when the outcome was important, we have tried to divide ourselves into smaller groups, so everyone could take part in the discussion. With these smaller groups, we set a time limit to come to a conclusion. This partly solved the problem, but it has still been an issue and it is something the group would have to continuously work on, if this company was to continue. 


The result from the SWOT analysis is somewhat similar to the Belbin test. They both explore the teams weaknesses and strengths, but they do it in different ways, and since they compliment each other, having both is a good idea.

\begin{table}[!ht]
\centering
\begin{tabular}{|c|c|}
\hline
\textbf{Strengths} & \textbf{Opportunities} \\ \hline
  \parbox[t]{0.45\textwidth} { %Strengths
  Our group contains people with experience with programming\\                            
  Our group is full of adaptable people with strong work ethics\\           
  We have experts on materials and sensors\\                                
  We can teach non-experts in the subjects we have an extensive knowledge\\ 
  } 
&
  \parbox[t]{0.45\textwidth} { %Oppertunities
  We can make this product without hiring external experts\\
  We will meet our deadlines\\
  We can work with different areas\\
  We will have a full knowledge of the project\\ 
  } \\ 
\hline
\textbf{Weaknesses} & \textbf{Threats} \\ \hline
  \parbox[t]{0.45\textwidth} { %Weaknesses
  We are not good at idea generation      \\
  We are not good at stopping discussions \\
  Loses focused when the goal is unclear  \\
  Other projects might distract           \\
  } 
&
  \parbox[t]{0.45\textwidth} { %Threats
  The project might get delayed\\
  We might focus on the wrong things\\
  We will have a lot of lose ends\\ 
  }\\ 
\hline
\end{tabular}
\caption{SWOT-analysis}
\label{SWOT-table}
\end{table}
