\subsection{Team SWOT Analysis} 
SWOT stands for: Strengths, Weaknesses, Oppertunities, and Threats. As the name says, it is a tool, which helps seeing what oppertunities and threats a team has, from its strengths and weaknesses. 
 
Table \ref{SWOT-table} shows the SWOT-analysis of the team. 
By looking at a SWOT analysis of each person we can determine what threats can be resolved by a another group member.
When there is a strength that is used to compensate for a weakness both are removed from the list. 
In the end we can make an overview of what skills we have as a team and what threats we have to deal with.

As a team, we are very good at solving and working with problems, but at the same time we are having difficulties finding and or creating these problems. 
This means that the team should be aware of difficulties, especially in the beginning of the project.

We are not good at ending discussions. We have tried to stop them after a certain time, and when the outcome was important, we have tried to divide ourselves into smaller groups, so everyone could take part in the discussion. With these smaller groups, we set a time limit to come to a conclusion. This helped the problem, but it has still been an issue and it is something the group would have to contuniously work on, if this company was to continue. 

The result from the SWOT analysis is somewhat similar to the Belbin test. They both explore the teams weaknesses and strengths, but they do it in different ways, and since they compliment eachother, having both is a good idea.

\begin{table}[!ht]
\centering
\begin{tabular}{|c|c|}
\hline
\textbf{Strengths} & \textbf{Opportunities} \\ \hline
  \parbox[t]{0.45\textwidth} { %Strengths
  Our group contains people with experience with programming\\                            
  Our group is full of adaptable people with strong work ethics\\           
  We have experts on materials and sensors\\                                
  We can teach non-experts in the subjects we have an extensive knowledge\\ 
  } 
&
  \parbox[t]{0.45\textwidth} { %Oppertunities
  We can make this product without hiring external experts\\
  We will meet our deadlines\\
  We can work with different areas\\
  We will have a full knowledge of the project\\ 
  } \\ 
\hline
\textbf{Weaknesses} & \textbf{Threats} \\ \hline
  \parbox[t]{0.45\textwidth} { %Weaknesses
  We are not good at idea generation      \\
  We are not good at stopping discussions \\
  Loses focused when the goal is unclear  \\
  Other projects might distract           \\
  } 
&
  \parbox[t]{0.45\textwidth} { %Threats
  The project might get delayed\\
  We might focus on the wrong things\\
  We will have a lot of lose ends\\ 
  }\\ 
\hline
\end{tabular}
\caption{SWOT-analysis}
\label{SWOT-table}
\end{table}

%% verbose swot table.. 
% \begin{table}[!ht]
% \centering
% \begin{tabular}{|p{0.45\textwidth}|p{0.45\textwidth}|}
% \hline
% \multicolumn{1}{|c|}{\textbf {Strength}} & \multicolumn{1}{c|}{\textbf{Opportunities}} \\ \hline
% Patient(2) & Problem solving(3)\\
% Tolerant & Good presenter\\
% Open minded & Broad contacts\\
% Working with others & Interested in management\\ %I am interested in management, so I can learn it for the project
% Communicative & Solve problems on time\\
% Strong work ethics (2) & Able to structure the report\\
% Adaptable & Can finish a project. \\
% Team player & Can work from somebody's schedule\\
% Open minded & Can work late\\
% Communicating & Not afraid to delegate and face impacts\\
% On time & Mindful of others and open for com-\\
% Social & munication for instance the workload\\
% & Can work in different areas\\
% Experience (work) & Idea generation\\
% Effective & Technical skills\\
% Technically skilled & Easily can learn other subjects\\
% Clever & Team worker\\
% Logic thinking &\\
% Able to prioritize &\\
% Well organized & \\
% Decisive(2) &\\
% Ambitious &\\
% Thorough &\\
% Decisive &\\
% Dedicated to solving issues/problem &\\
% Comprehensive &\\
% Discipline on my own &\\ %Discipline when it comes to working on my own
% Creativity and innovation &\\
% \hline
% \multicolumn{1}{|c|}{\textbf {Weaknesses}} & \multicolumn{1}{c|}{\textbf{Threats}} \\ \hline
% Stubborn(2) & Not a specialist\\
% Impatient & Easily get stressed\\
% Inflexible & Impatient, if others don't understand\\ %Have a hard time when there is something others do not understand
% Being on time & Might be difficult to understand\\
% Express my ideas & Bad at solving problems myself\\
% Unwilling to recognize the value of my & Bad at remembering details\\ %Bad at remembering details, can have an impact on the overview as a whole
% work & Might ignore good suggestions when focused on other/own ideas\\
% % It is the point that there i a space between [work] and [Not starter (If goal is unclear)(2)]
% Not starter (If goal is unclear)(2) & The development phase might be slowed \\
% Loses focus easily(3) & down\\ %2nd part is from line above
% Not very innovative/creative (2) & Reduced working time \\
% Overview & I like parties and going out/I prefer fun \\
% Working fully on my own & over work\\ %2nd part is from line above
% Skeptical within my area & Focus on too many areas\\
% Not a perfectionist & We might never get started\\
% Bad at keeping track of who knows what & Bad at getting ideas to startup a project\\ %Not good at keeping track of who knows what
% Meeting deadlines & Losing focus\\
% Uncomfortable with uncertainty & Need things planned in good time\\
% & Get stalled in some point of the project\\
% & Don't finish on time\\
% \hline
% \end{tabular}
% \caption{SWOT-analysis}
% \label{SWOT-table}
% \end{table}