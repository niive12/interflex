\subsection{Conclusion}
The team has had a lot of difficulties finding a problem that we wanted to work with. We have been using innovative tools including brainstorming to come up with ideas particularly around e-waste but we never got anything useful. After a meeting with the supervisors who told us that we needed to move forward, we decided to work with an idea that was mentioned in the introduction of the project.

By looking at the results from the Belbin and SWOT-table it is not surprising that the team ended in the situaton that we did. It is apparently very clear that the team has a weakness when it comes to idea generation and as well a strength in problem solving. Prospectively it would be a good idea to look at the results from the team tests so we don't end up in the same situation as we already have.
%Vi har kæmpet længe med at finde det problem vi vil arbejde med, i projektet. Vi har brainstormet meget omkring affald, i sær e-affald men kom aldrig frem til noget brugbart. Efter at vores vejledere sagde at vi skulle til at komme videre, fik vi ændre projekt ide til en ide der er lagt lidt op til i projekt introduktionen.

%Ser vi på vores resultat fra Belbin og SWOT-tabellen er det ikke overraskende at vi havnede i den situation som vi gjorde. Det fremgår nemlig meget tydeligt at vi har en styrke i problemløsning men tillige en svaghed i idégenerering.Det vil fremadrettet være en god idé at se på vores styrker og svagheder fra de team tests vi har foretaget.
