\subsection{Analysis Tools}
To measure participation and contribution, different analysis tools becomes viable. These tools provide an overview of each members activity during timespan of the project. Two tools have been used during the process, which are described in the sections below.

\subsubsection{Group Activity Impact Tool}
 The Group Activity Impact Tool (GAIT) is used for monitoring the relative workload of all members of a team. This will assist in evenly distributing the workload, possibly increasing the overall productivity of the team. Additionally, the tool exposes any team members that are less productive than what is expected. Every task has to be manually weighted to reflect the difficulty and workload of the task. If the difficulty or workload of a task is under- or overestimated the reflection of the workload of the members given by the tool will not be correct.
This tool has been used only sparingly throughout the course of the project. Very often it has been hard to identify individual tasks and harder still to determine the importance/weight of each task. These difficulties meant that any benefits of distributing tasks using the tool did not compensate for the extra time associated with using the tool. 

By not using any structured method of distributing tasks, other than identifying tasks (where possible) and people volunteering for any task that they find interesting, we may have skewed the relative workload of the members. Certainly, it has meant that some people have worked almost exclusively on one area within the project. The incomplete GAIT table can be seen in the figure \ref{fig:GAIT}.

\begin{figure}[h!]
	\includegraphics[width=\textwidth]{./graphics/gait}
	\caption{The GAIT tool with tasks made by the group}
	\label{fig:GAIT}
\end{figure}

\subsubsection{Group Participation Impact Tool}
The Group Participation Impact Tool (GPIT) is simply a tool for monitoring the meeting participation of each member. It provides a graphical representation of the participation. Using this tool will provide the team with concrete evidence of the participation of each member. This could prove useful, should a situation, where a team members contribution is in question, arise. This tool was used throughout the first month of the project but was since neglected. The results can be seen in figures \ref{fig:GPIT} and \ref{fig:GPITGraph}. Since the GPIT has only been filled out the first month, obviously, it does not give a complete image of the participation. Furthermore, according to the group contract, people agreed to working eight hours a week. These hours were not necessarily restricted to the meetings and as such this tool may not have provided us with a true representation of the actual contribution.

\todo[inline]{Update GPIT picture with names. Consider removing this altogether - It does not say anything meaningful.}

\begin{figure}[h!]
	\includegraphics[width=\textwidth]{./graphics/gpit_attendance}
	\caption[The GPIT tool]{The GPIT tool keeps track of attendance. If the person was there on the given date, they are given a 1.}
	\label{fig:GPIT}
\end{figure}

\begin{figure}[h!]
	\includegraphics[width=\textwidth]{./graphics/GPIT_impact}
	\caption[The contribution by each member]{The contribution to the project by each member according to the GAIT/GPIT Tools}
	\label{fig:GPITGraph}
\end{figure}