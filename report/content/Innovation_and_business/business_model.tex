\subsection{Business Model Creator (IDEA - BMC)}
\begin{figure}
%   \includegraphics[width = 0.5\textwidth]{./graphics/} %what picture did you want to include?

\end{figure}

After defining the business idea we wanted to specify problems that our idea could solve. 
To get the maximum from our idea, we decided to make a business plan. 
The question was how to generate cash flow and to create value for the customers and this helped us answer it. 
It helped us to better define the situation in which we found ourselves and the direction we should take.

We started with “Value Proposition”, which should help us define the services we would like to provide and the products we will develop the business model for. 
We saw this step as very important and that it will define the course of our business plan, so we decided to spend some time thinking about how we could define what we are going to do and the value we want to create for potential customers.

In the end, we defined our customers as the companies in the welding industry. 
Companies working on improving the welding technology and companies with non-mass production. 
We expect the first to be interested in acquiring the technology to implement it to their own system to gain a competitive advantage and the latter one can use it to improve their production and facilitate flexibility of the production, making it possible for further customization of their products and reduction of costs and time.

In the next step, we began to define ideas related to the product.
We saw an opportunity in creating value from our product as an innovation in areas, where the competitors failed to achieve it. 
We decided that the best option in terms of price of the final product would be  to suit the current market price, as it would be almost impossible to sell it under the price and selling it with a high price would not be profitable for our customers as the value it creates for them is not crucial for their production.

Will have to make the following comments in reference to the product configuration:

\begin{itemize}
\item We are trusted partner in a highly integrated value chain. We focus on adding value in a very specific chain.
\item As we focus on developing the technology necessary for the development of sensing system lines for automation of welding, a strong relation with our partners will be necessary, as we need the rest of the technology and components, in order to create the full product.
\item Our processes will be quite the same as the industrial production in general:
\begin{itemize}
\item Inbound logistics
\item Production
\item Outgoing logistics
\item Sales and marketing
\end{itemize}
\end{itemize}

There is also the financial part. 
Our prices depend on the product features. 
The more or the better the features, the higher the price, so it will be more expensive if we had to develop a new type of product with different specifications than if they buy the standard product. We will try to make a price list suitable to all kinds of potential customer's production.

Finally, we have the customer configuration table \ref{tab_conf}. 
We have an extremely narrow area of focus, but we can develop the product in response to the customer needs, which we know. 
As this is not a cheap product and the market is not that big, we will try to keep our customers through loyalty programs, where customers are rewarded for remaining loyal to our product. 
This is also a good way to acquire new customers, because if they are happy with the service/product we provide them, it is very likely that they will provide positive references and recommendations and spread the message about our product.

\begin{table}[ht]
\centering
\begin{tabular}{|m{2cm}|m{2cm}|m{2cm}|m{2cm}|m{2cm}|m{0.1cm}}%for some reason does the last cell not follow this rule so I added an invisible cell...
\rowcolor{Green}
\multicolumn{5}{l}{Customer Configuration}\\
\rowcolor{Green}
\multicolumn{2}{l}{Channel}\\
\cline{1-5}
Channel            &  Awareness        &    Evaluation    &     Purchase     &   After Sales   &\\\cline{1-5}
Internet           &  \(\mycheckmark\) & \(\mycheckmark\) & \(\mycheckmark\) & \(\mycheckmark\)&\\[1cm]\cline{1-5}                                                                       
Product brochures  &                   & \(\mycheckmark\) &                  &                 &\\[1cm]\cline{1-5}
Journals           &  \(\mycheckmark\) &                  &                  &                 &\\[1cm]\cline{1-5}
\end{tabular}
\caption{Customer configuration \label{tab_conf}}
\end{table}
