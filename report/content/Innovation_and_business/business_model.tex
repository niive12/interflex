\subsection{Business Model Creator (IDEA - BMC)}
\begin{figure}
%   \includegraphics[width = 0.5\textwidth]{./graphics/} %what picture did you want to include?

\end{figure}

After defining the business idea we wanted to specify problems that our idea could solve. 
To get the maximum potential from our idea, a business plan will be created. 
Creating a business plan will help in answering the question of how to generate a positive cashflow as well as generating value for customers. 
Additionally, it will help us defining our current situation and deciding which direction to take. The following sections explain the steps that were taken in developing the business model for our company.

\subsubsection{Value Proposition}
The Value proposition is a crucial part of the business model and should help us define both the services and the product that we would like create a business model for. 
Since this part could potentially assist in devoloping our project, it was chosen to start with this part.

Our expected customer base are companies in the manufacturing industry which utilize welding as part of their non-mass production chain. 
Additionally we expect to be able to sell our technology to companies which supply the industry with welding equipment. 
A company that decides to implement our technology will see a significant increase in flexibility and as such, gain an advantage in an increasingly competitive market. 
The increased flexibility will allow for futher customization of products and a reduction in both manufacturing costs and time.

It is important to us that our product is economically viable for our customers and as such we did research to determine whether there is room for a product like ours on the market.
Not surprisingly, there is a great demand for flexibility, especially amongst smaller production companies. 
We believe that our product can provide this flexibility.
However, for SME's cost is everything. By pricing our product to suit the current market price of other welding solutions, we ensure that it is seen as a viable, competitive option for SME's.

Will have to make the following comments in reference to the product configuration:
\todo[inline]{What?}

\begin{itemize}
\item We are a trusted partner in a highly integrated value chain. Our focus lies on adding value in an already established market.
\item \todo[inline]{This as alternative to next point} In order to always have the latest technology available to aid in devoloping sensory equipment for automation of welding robots, it is important that we maintain a strong relation with our partners.
\item As we focus on developing the technology necessary for the development of sensing system lines for automation of welding, a strong relation with our partners will be necessary, as we need the rest of the technology and components, in order to create the full product.
\todo[inline]{Our processes.. What does this mean? Is it significant?}
\item Our processes will be quite the same as the industrial production in general:
\begin{itemize}
\item Inbound logistics
\item Production
\item Outgoing logistics
\item Sales and marketing
\end{itemize}
\end{itemize}

\subsubsection{Finances} 
\todo[inline]{Does this mean that we have to define different products with different specifications?!}
The price of the product depends highly on the included features. 
The more or the better the features, the higher the price. Naturally, the product will be more expensive if we have to develop a new type of product with different specifications than if they buy the standard product. We will make a price list suitable to all kinds of potential customer's production.
\todo[inline]{This seems a tad short for such a huge part of a company?}
\subsubsection{Customer Configuration Table}
\todo[inline]{Short: what is this why is it important}
Finally, we have the customer configuration table as seen in table \ref{tab_conf}. 
We have a narrow area of focus, but this means that we can develop the product in response to customer needs.
\todo[inline]{Does the next part make sense when our product is distributed through... distributers? Also, don't tell people our product isn't cheap - you'll never hear apple say their products are expensive..}
As this is not a cheap product and the market is not that big, we will try to keep our customers through loyalty programs, where customers are rewarded for remaining loyal to our product. 
We believe that keeping customers happy with our products and services will assist in acquiring new customers through word of mouth. Even if our customers would rather that their competition will not gain part in the advantage that they have gotten in our product, hiding a success will not be possible for long.

\todo[inline]{Make sure this table matches the rest - potentially make a pdf from excel. urrghh..}
\begin{table}[ht]
\centering
\begin{tabular}{|m{2cm}|m{2cm}|m{2cm}|m{2cm}|m{2cm}|m{0.1cm}}%for some reason does the last cell not follow this rule so I added an invisible cell...
\rowcolor{Green}
\multicolumn{5}{l}{Customer Configuration}\\
\rowcolor{Green}
\multicolumn{2}{l}{Channel}\\
\cline{1-5}
Channel            &  Awareness        &    Evaluation    &     Purchase     &   After Sales   &\\\cline{1-5}
Internet           &  \(\mycheckmark\) & \(\mycheckmark\) & \(\mycheckmark\) & \(\mycheckmark\)&\\[1cm]\cline{1-5}                                                                       
Product brochures  &                   & \(\mycheckmark\) &                  &                 &\\[1cm]\cline{1-5}
Journals           &  \(\mycheckmark\) &                  &                  &                 &\\[1cm]\cline{1-5}
\end{tabular}
\caption{Customer configuration \label{tab_conf}}
\end{table}
