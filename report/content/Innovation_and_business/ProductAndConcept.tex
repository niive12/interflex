\documentclass[11pt,a4paper]{article}
\usepackage[utf8]{inputenc}
\usepackage[english]{babel}
\usepackage{amsmath}
\usepackage{amsfonts}
\usepackage{amssymb}
\usepackage{graphicx}
\usepackage{float}
\usepackage{picture}
\usepackage{fullpage}

\author{Thomas Søndergaard Christensen}
\title{---}
\date{13/11/2014}

\begin{document}
\section{Product and Concept}
\subsection{Customer Value}
The general idea of a welding robot is valuable in the way that no worker has to do the welding. This means that no one has to be close to the welding which can be hazardous. It also means that the welding done will be more precise, since the robot does not get tired or unfocused. All these qualities will of course also be in a robot. The part we want to add to the robot  makes sure that there is no programmer needed for each welding, thereby the user of this product will save labor. Besides that, it is very flexible, and can weld a different thing each time, without the preparation being longer.
\subsection{Core Product}
We propose a tool which can be attached to and interface with an existing welding robot. 
A worker using a special marker will mark the area on a product that requires welding. By combining computer vision and sensors the tool will enable the robot to automatically locate and weld the seam. This scheme will completely eliminate the need for expensive and time consuming programming processes.

\subsection{Pricing}
A modern welding robot system will cost around one mio. DKK, the cost of our tool will be added to this price. In order to stay competitive with the programming solutions currently in existence, it is important that the added cost is kept low while still maintaining the quality of the work that the robot can do. Keeping both competition and quality in mind, material and manufacturing costs are close to 30.000 DKK\footnote{A breakdown of the price can be seen in appendix [Price of tool]} The final sales price is therefore set to 50.000 DKK.
\subsection{Development Potential}
The first product is for welding on new constructions. It can be used for welding different sizes, limited by the robot arm it is attached to. Developments of this product will also be able to weld on broken products. Here the advantages of not having to program will truly show, since no two damages are exactly the same.
\subsection{Production}
For this product already invented technologies are put together in a new way. Therefore it is produced by ordering the different parts and putting them together. The only thing needed then a shell for the parts, which will be custom made once it is known how big the product will be. Once the product becomes popular it will be able to also assemble it in a country with cheaper labor. Since the product is relatively small, it will be easy and cheap to ship it to the location where it is needed.
\end{document}