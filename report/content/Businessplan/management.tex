\section{Management and organization}
The concept is an equally owned idea between the group members. It has been declared since the beginning of the project in the group contract.
Since the group consist of students, the project will require funding to become a reality. No one is interested in being personally liable for the debt if the company should go bankrupt. That is why creating an ApS would be a good option. This requires at least 50.000 kr. of value (can be values of objects too) to settle. The risk of doing it this way is that the investors will probably force the group members to sign for a personal liable agreement. 
When investors act, the ownership will probably change also as they will be a part of it.

\subsection{Legal structure and ownership}
The concept is an equally owned idea between the group members. It has been declared since the beginning of the project in the group contract.
Since the group consist of students, the project will require funding to become a reality. No one is interested in being personally liable for the debt if the company should go bankrupt. That is why creating an ApS would be a good option. This requires at least 50.000 kr. of value (can be values of objects too) to settle. The risk of doing it this way is that the investors will probably force the group members to sign for a personal liable agreement. 
When investors act, the ownership will probably change also as they will be a part of it.
\todo[inline, color=red!50]{Going once, Going twice...}

\subsection{Management}
A conclusion has been drawn from the Belbin group result. It is based on the requirements of the different titles within the company.  For instance, we felt that the CEO would require a strong represent who has a great overview (Acts as a coordinator), a talent for communication (Resource Investigator) and the drive of a shaper. Within these three fields, a score was created based on the individual Belbin results, which lead to choosing David as the CEO. 
The same procedure was followed when assigning candidates for the other posts:
\begin{itemize}
\item CEO, David
\item CTO, Xabier
\item CFO, Casper
\end{itemize}

\todo[inline, color=red!50]{MISSING HIERARCHY STRUCTURE OF THE COMPANY} 

\subsection{Board and advisors}
As usual, the leaders of a company will be sitting within the board, but here by different partners would also play a big role within. If KJV decides to sell our product, they could have an interest in driving our company in certain directions to increase the sale. This would benefit both of us. 
It could be interesting to have other partners within the board, especially some within the technological field, robot- and software wise.
One of the most valuable advisors to have would be a person with experience within innovation and technology, who also has experience with a start up company. 
\todo[inline, color=red!50]{Casper: is it still KJV}
\subsection{Partnerships}
The strategy of the company relies on having different distributors within Europe, therefore it is essential to have partnerships with these companies. They are not the final customer of our company, but a link to them. This is where the main revenue streams is going to be created.
For instance, KJV would be an optimal partner within the Danish market, since they are able to reach the final customers within Denmark. Distributors like KJV is our goal to reach within the market of Europe.
Another sort of interesting partners is one of the majors of the market. This could be Migatronic, Valk Welding, Universal Robots or any other sort of major company within the field of technology, who already exist on the market. The reason for this is that they could be interested in the technology our product offers and that they would like to integrate it within their range of products. This situation would usually lead to them investing in our company and putting a limit to which companies we are allowed to collaborate with.

\subsection{Key activities}
For the project to become a reality it would require financing, unless the group members settles for working on the project through a period of their masters. This would result in a different direction of development, than with a basis capital. Because software is doable without the need of an investment, but when it comes to combining the hardware and software for testing purposes, it is going to require an injection of funds. In this case, the obvious choice would be the group trying to find investors.

Therefore the development of concept is crucial for the company while pitching the idea for potential investors. Other key activities will include the process of combining software and hardware and further testing to improve the product. 

\subsection{Key resources}
The key resources are the assets to the company that are necessary to create value for the customer. These shall support the company and make it sustainable.
To summarize the primary key resources to the company:
\begin{itemize}
\item Injection of funds
\item Accommodations for office and development usage
\item Consultancy in form of business innovators who can advice and share experiences
\end{itemize}

\todo[inline, color=red!50]{Casper: We could divide into steps.. Development phase, productions pahse..?}