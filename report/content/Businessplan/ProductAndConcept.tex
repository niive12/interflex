\subsection{Product and Concept}
\thomas{Again, Lots of information here that is repeated several times throughout the report}
\subsubsection{Customer Value}
A welding robot will, on average, replace 4-5 human welders, significantly decreasing the cost of wages. Additionally, robots can alleviate workers of potentially dangerous tasks. The quality of the welding done by robots is not only more consistent than that done by humans, who might tire or loose focus, it is also of a higher quality. The tool that we wish to add to the robot will further decrease the labour needed to operate a manufacturing line by removing the need for a programmer. The workers will be able to quickly and easily instruct the robot in the job at hand. This will significantly increase flexibility and potentially allow for smaller businesses to make an otherwise impossible investment.

\subsubsection{Core Product}
We propose a tool which can be attached to and interface with an existing welding robot. 
A worker will, using a special marker, mark the area on a product that requires welding. By combining computer vision and sensors the tool will enable the robot to automatically locate and weld a seam. This scheme will completely eliminate the need for the expensive and time consuming programming processes.

\subsubsection{Pricing}
A modern welding robot system will cost around one mio. Dkk, the cost of our tool will be added to this price. In order to stay competitive with the programming solutions currently on the market, it is important that the added cost is kept low while still maintaining the quality of the work that the robot can do. Keeping both competition and quality in mind, material and manufacturing costs are an estimated 40.900 Dkk\footnote{A breakdown of the price can be seen in appendix \ref{app:priceofProduct}} 
The final sales price is therefore set to 80.000 Dkk.

\subsubsection{Development Potential}
The initial iteration of the product is designed for producing new products. It is limited only by the size of the welding robot that it is attached to. We envision a future for our product in repair and maintenance of existing products. This is a field where models and standardized methods are rarely in place, and the strengths of our programmer-less approach will truly show its usefulness.

With few changes the product could be altered to be able to guide a robot to repaint cars or perhaps even ships. It may be feasible to use it for otherwise dangerous cutting tasks using a blowtorch. In short, the potential is vast.

\subsubsection{Production}
An important aspect of this product is that costs must be kept low. One way of doing this is the use of existing technologies to achieve something new. Production is as advanced as ordering parts and assembling them. The only custom part needed in the product is the housing, which will be ordered from a machinist. Initially, assembly will be done nationally, and shipped internationally. Shipping is not an issue due to the limited size of the product. Once profit allows it, it will be considered whether production should be moved to countries with lower production costs.
