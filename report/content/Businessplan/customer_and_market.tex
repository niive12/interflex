\subsection{Customer and Market}
% In order for our project to be successful it is important that the customer base is well defined. This will help tailoring the product to the customer's needs. This section will explain which market this product is aimed at and the reasons for this choice. 

Defining the customers and the end users is important to shape the goals of our company. 
By tailoring the product to the customer's needs we can increase our potential and maximize our profit. 
By considering the market we can learn from our competitors and by monitoring the current trends in the market we can see if our idea is going to work in the market. 

\subsubsection{Expected Customer Base}
Our direct customers are the companies that sell complete welding system solutions to production companies. 
These companies will already have contacts within the customer base that our products targets.
They will know their needs and how to reach them. 

We want to make it easy for distributers to package our solution with their existing products.
To do this we will adapt our solution to be able to work with existing welding robots.

We can add value to our customers by letting them expand to business that requires a more flexible solution.

\subsubsection{End users}
The end users will be small to medium sized enterprises [SME's] in the metal working industry, specifically companies with many small, different production runs that would benefit greatly from highly flexible solutions.
Since we don't have a direct relation with the end users we get information about their needs from our distribution partners.
The end user want to spend the smallest amount of time on programming and the highest amount of time welding with their robots.
We want to make robot programming more intuitive to achieve these demands.

\subsubsection{Market}
Initially we want to focus on distributors located in Denmark.
This makes logistical issues easier to deal with and until we have a product that we can send out everywhere we want to work closely with our customers.

One of our first customers would be Valk Welding. They create welding systems based on Panasonic robots. They try to create programs that makes easier the welding process, minimizing the time required to program the robot.
\casper{the sentence above}
The customers of Valk Welding are mainly using offline programming.
With a complete knowledge of the item that should be welded and the positioning of the robot they can program every detail of the welding process.
The programming is made in a very high level language that keeps track of the coordinates, the speed and the angles of the weld.
With macros that aid the programmer making the same routines in different places the time it takes to program the robot is drastically reduced.

We want to be able to replace the training course required to learn how to do offline programming and the end customer will save money by having a smaller staff and, potentially, producing more items. 
Currently a training course in online programming costs around 100,000 DKK\cite{valk_welding_summary}. 
The average yearly pay of a welding robot programmer 370,000 DKK\cite{welding_salary}. 

Another potential customer would be the company Weld-Tech ApS. 
This company is already working with automatic welding, but currently has the restriction that all items need to have identical patterns. 
This might hinder their ability to attract new customers. An issue that we may be able to aleviate using our technology.

When the product is finished we want to expand to be able to use this technology for bigger and more complex welding jobs.

\subsubsection{Trends}
\label{sec:trends}
The production industry is trending towards more and more automation.
Companies invest heavily in new technologies to gain a competitive advantage. 
The two main programming paradigms for robots in use today are:
\begin{enumerate}
\item \textbf{Online Programming:} This paradigm depends on being able to guide the robot throughout all the necessary welding seams. As a result, the robot will be unable to run production for the full length of the programming cycle.
\item \textbf{Offline Programming:} This method is the preferred method in use today. It requires that a CAD model of the product exists. 
The programming of the robot is then done using the CAD model. 
This means that during the entire programming process the robot can still work on other products and only has to be interrupted when the program is ready to be put in production. 
Not only does offline programming require much less downtime of the robot, the programming itself is also 5-10 times faster than the conventional online programming scheme.	
\end{enumerate}
The field of automatic programming is still in development and many companies create prototypes. We believe that we can add significant progress to a third paradigm that will expand the use of robot welders to smaller businesses and productions:
\begin{enumerate}
	\item[3.]{\textbf{Off-site Programming:}} This paradigm is currently in development and holds great promise for especially one-off productions. A program for a welding robot is made using some sort of teaching-mechanism. 
	This way, any worker with knowledge of the product in question can program the robot.
\end{enumerate}
