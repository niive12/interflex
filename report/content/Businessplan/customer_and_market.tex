\subsection{Customer and market}

\subsubsection{Expected Customer Base}
Our direct customers are the companies that package and sell welding systems to other companies. 
These companies are responsible for getting in contact with the end users. 
They will know their needs and how to reach them. 
These users tend to be companies that produce small to medium batches, which typically have experience with the automation of welding robots.

We want to make it easy for our customers to package our solution with their existing products.
To do this we will adapt our solution to be able to work with existing welding robots.

We can add value to our customers by letting them expand to business that requires a more flexible solution.

Despite all robots differences in protocol, the basic principal stays the same.
To make a product as soon as possible we would start working with a single company so there is only one type of robot programming protocol.

\subsubsection{End users}
Our end users is the our customers clients and a lot of our considerations.
Since we don't have a direct relation with the end users we get information about their needs from our customers.
The end user want to spend the smallest amount of time on programming and the highest amount of welding time out of their robots.
We want to make robot programming more intuitive to achieve these demands.

\subsubsection{Market}
Initially we want to focus on distributors located in Denmark.
This makes logistical issues easier to deal with and until we have a product that we can send out everywhere we want to work closely with our customers.

One of our first customers would be Valk Welding. They create welding systems based on Panasonic robots. They try to create programs that makes easier the welding process, minimizing the time required to program the robot.

Valk weldings customers is mainly using offline programming.
With a complete knowledge of the item that should be welded and the positioning of the robot they can program the way an item should be welded.
The programming is made in a very high level language that keeps track of the coordinates, the speed and the angles of the weld.
With macros that aid the programmer making the same routines in different places the time it takes to program the robot is drastically reduced.
We want to be able to replace the training course required to learn how to do offline programmeing and the end customer will save money by having a smaller staff and, potentially, producing more items. 
Currently a training course in online programming costs around 100,000 DKK\cite{valk_welding_summary}. 
The average yearly pay of a welding robot programmer 370,000 DKK\cite{welding_salary}. 

Another potential customer would be the company Weld-Tech ApS. 
This company is already working with automatic welding, but currently has the restriction that all items need to have identical patterns. 
This might hinder their ability to attract new customers. an issue that we may be able to aleviate using our technology.

When the product is finished we want to expand to be able to use this technology for bigger and more complex welding jobs.

\nikolaj{Make a better valk meeting summary}

\subsubsection{Trends}
The trends in the welding industry are constantly changing. 
Companies invest heavily in new technologies to gain a competitive advantage. 
This means that there are always incorporating new ideas to this sector, but we must always look at the cost stemming and the final price of the product, as SMEs are not willing to pay much for slightly improve welding method.
\nikolaj{We need some kind of source on this}

There are two programming paradigms that are used today.

One method is online programming, where programming is done by physically moving the robot around and logging the points it has to move. For a complex product, this may take up to 3 or 4 weeks.

Another method is offline programming where programming is done my making a program around a CAD model of the product.
It usually takes 2 to 3 days to make a similar program.

The field of automatic programming is still in development and many companies create their prototypes. 
What stops these products in becoming popular are their tradeoffs with a higher price without adding any real flexibility.
\thomas{Do we have reason to believe that this is the case?}