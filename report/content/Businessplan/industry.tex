\subsection{Industry structure and environment}
\label{ind.struc}
A meeting with Valk Welding in Nr. Åby\cite{valk_welding_summary} gave us some insight in the robotic welding industry. Valk Welding sell complete robot welding solutions from Panasonic but with their own modified software. The market had a total sale of 22 units across the industry in 2013 in Denmark. Figure \ref{competitors} shows other suppliers of robot welding systems currently operating. Valk Welding said that they did not do canvassing, the production companies came by themselves. Valk Welding offer two kind of programming solutions an online and an offline. 
Online programming requires the programmer to guide the robot through all of the necessary steps. This will typically take 5-10 times longer than offline programming. Additionally the robot will not be available for use while it is being programmed. The offline programming scheme requires a programmer to define all of the necessary steps on a 3D model. Once completed, the program is tested on the robot to eliminate any faults made by the programmer. The latter is currently the most sought after solution since it provides the highest flexibility and requires the least amount of downtime.

At a robot exhibition in Copenhagen the 17th of November 2014 Valk Welding announced their "pistol\footnote{Looks like the last joint of a welding robot}" for robotic welding. Within a 3D camera zone you take the pistol and place it where and how you want to weld, click it and then place it where the welding should end and click it\footnote{Only works in a straight line though}. Then the robot welds the marked area. This is a slower process than the offline programming on larger welding tasks.
\subsection{Entry barriers}
There are several entry barriers to the market. The biggest barrier is that we enter a market which already have a firm supply chain in place. There are companies which have years of experience in dealing with the demands and behaviours of the customers. Although we do supply something new to the market, persuading customers that our product is superior to, who might already have tried and tested other solutions, may be difficult.

Additionally it is crucial that our product follows the KISS-principle\footnote{KISS: Keep It Simple Stupid}. It is our intention that the product will simplify the welding process, however, adding another potential link of failure to a system may not be desirable.

Compared to other companies we may have trouble keeping up with development due to the limited R\&D-budget available.
\subsection{Competitors}
\label{competitors}
All manufacturers of robot welders are more or less competitors or potential partners/buyers. They too, are working on more flexible solutions for customers. The easier it is to program your robot, the more flexible it is. The main competitors are listed in table\ref{Tablecompetitors}. The resellers Migatronic and Valk Welding buy or cooperate with the robot manufactures ABB and Panasonic, respectively. All the listed competitors are on the same level of competition.

\begin{table}[h]
\centering
\begin{tabular}{|l|c|c|}
\hline
             & Welding robot resellers & Welding robot manufactures \\ 
\hline
Valk Welding & X                       &  \\ 
\hline
Panasonic    &                         & X \\ 
\hline
Migatronic   & X                       &  \\ 
\hline
Kuka         &                         & X \\ 
\hline                                 
Fanuc        &                         & X \\ 
\hline                                 
ABB          &                         & X \\ 
\hline                                 
Yaskawa      &                         & X \\ 
\hline
\end{tabular} 
\caption{Brief overview of significant resellers and producers of welding robots}
\label{Tablecompetitors}
\end{table}

\subsection{Competitive Advantage and Strategy}
As previously mentioned the main demand from SME's when investing in welding robots today, is increased flexibility.
 The risk of investing in a welding system decreases with higher flexibility. This is where our product will gain a competitive edge over other solutions. We can completely eliminate the need for expensive and time consuming programming. Any worker with knowledge of welding will be capable of instructing the robot with minimal training. This will allow for quick transition between production of different products. Basically, our product will provide companies with currently unmatched flexibility, making the transition to an automized production line viable for much smaller companies than previously.