\subsection{Industry structure and environment}
\label{ind.struc}
A meeting with Valk Welding in Nr. Åby gave us some insight in the robotic welding industry. Valk Welding sell total robotic welding solutions from Panasonic but with their own modified software. The market had  a total sale of 22 units across the industry in 2013 in Denmark and with a few competitors\footnote{See \ref{competitors}} the market seems pretty tough. Valk Welding said that they did not do canvassing, the production companies came by themselves. Valk Welding offer two kind of programming solutions an online an a offline. The online solution means that the robot is programmed with the controller that is connected to the robot, so you can see how the robot is moving while programmed. This means that the robot can't work while being programmed and this can take up to several weeks. The offline solution takes place in a software program where a 3D-drawing of the object is uploaded. Here the programmer can program the robot while it is working. It takes 5-10 times as less time to program offline compared to online. The demand on the market is offline programming solutions because it gives a lot of flexibility.

A at robot exhibition in Copenhagen the 17th of November 2014 Valk Welding announced their "pistol\footnote{Looks like the last joint of a welding robot}" for robotic welding. Within a 3D camera zone you take the pistol and place it where and how you want to weld, click it and then place it where the welding should end and click it\footnote{Only works in a straight line though}. Then the robot welds the marked area. This is though a slower process than the offline programming if a lot of welds is needed.
\subsection{Entry barriers}
There are several entry barriers on the market. A big one is to compete with companies already in the industry which has lot of years of experience and insight in customer demand and behavior. These companies are already working on flexible solutions, which is the marked demand. 
\subsection{Competitors}
\label{competitors}
All manufactures of robotic welders is more or less a competitor or a potential partner/buyer. This is because they are all working on more flexible solutions for the customer. The easier it is to program your robot, the more flexible it is. The main competitors is listed in table\ref{Tablecompetitors}. The resellers Migatronic and Valk Welding buy or cooperate with the  robot manufactures ABB and Panasonic, respectively. All the listed competitors are on the same level of competition.

\begin{table}[h]
\centering
\begin{tabular}{|l|c|c|}
\hline
             & Welding robot resellers & Welding robot manufactures \\ 
\hline
Valk Welding & X                       &  \\ 
\hline
Panasonic    &                         & X \\ 
\hline
Migatronic   & X                       &  \\ 
\hline
Kuka         &                         & X \\ 
\hline                                 
Fanuc        &                         & X \\ 
\hline                                 
ABB          &                         & X \\ 
\hline                                 
Yaskawa      &                         & X \\ 
\hline
\end{tabular} 
\caption{List of most known resellers and producers of welding robots}
\label{Tablecompetitors}
\end{table}


\subsection{Competitive advantage and strategy}
As previously mentioned the main demand from SME's when investing in welding robots today, is increased flexibility.
 The risk of investing in a welding system decreases with higher flexibility. This is where our product will gain a competitive edge over other solutions. We can completely eliminate the need for expensive and time consuming programming. Any worker with knowledge of welding will be capable of instructing the robot with minimal training. This will allow for quick transition between production of different products. Basically, our product will provide companies with currently unmatched flexibility, making the transition to an automized production line viable for much smaller companies than previously.
\thomas{How do you like them apples?}