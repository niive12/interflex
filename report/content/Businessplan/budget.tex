\subsection{Budget}
\label{budget_label}
The budget has been divided into a development budget and a operation budget. The development budget \todo[inline]{Budget: ref to full budget in appendix} shows how much money we need for our concept to be ready for production. We assume that if we don't get any funding then the bank loan during the development phase will be with exempt from payments until we start production. The main costs of the development budget is the wages, establishing, rent and some unexpected costs. 
The basis: \begin{itemize}
\item[-] for the wage is that the team (7 persons) will be working full-time (160,33 h/month) to a wage equal to the minimum salary for a third year student (172 DDK/h\cite{ida-salary}).
\item[-] for establishing is covering office fittings (37000 DKK) and application for a patent/registration.
\item[-] for rent is a price check for renting commercial premises in Odense, Denmark. Premises with plenty of space is available at 5000 DKK.\cite{rent_prices}
\item[-] for the interest of the bank loan of 7 \% is that the concept is with some risks \ref{risk} , if there were no risks the interest would be lower. 
\end{itemize} 
The cost are listed in table \ref{devbud}, a full description can be found in appendix \todo[inline]{Budget: ref of description}. As it appears in the table we need 2.706.470 DKK to develop our concept. If no funding is granted a bank loan of 2.750.000 should be sufficient.

\begin{table}[h!]
\centering
\begin{tabular}{l r}
Development budget      & (DKK)\\
\hline                  
Variable costs:         &  \\
Bill of materials       &  28.500  \\
Regular costs:          &  \\
Wage                    &  2.316.448\\
Rent                    &  75.000\\
Establishment           &  87.000\\
Others                  &  79.000\\
Unexpected costs (5\%)  &  123.522\\
\hline                    
Total                   &  2.709.470\\
\end{tabular}
\caption{Summary of development budget}
\label{devbud}
\end{table}

The operation budget shows the cash-flow when we start producing and selling our product. We have made a budget for the first two years after the development phase. The operation budget consist of turnover, variable costs, regular costs and the bank loan if no funding is provided. We have made the following assumptions:
\begin{itemize}
\item[-] The first year we guess to around 60 units and the next year around three times as much.
\item[-] Bill of materials is proportional with units sold.
\item[-] Regular cost stays the same as during the development phase.
\item[-] 7 percent in annual interest on bank loan because there is some risks. The payback is 40.000 per month. This gives a payback time on approximately 6 years.
\end{itemize}
A summery of the operation budget is shown in table \ref{opebud}, a full description can be found in appendix \todo[inline]{Budget: ref}.
\begin{table}[h!]
\centering
\begin{tabular}{l r r r}
Operation budget      &           &           & (DKK)      \\
                      & year 1    & year 2    & year 3     \\
\hline                
Turnover              &         0 & 4.720.000 & 14.000.000 \\
Variable costs        &    28.500 & 2.412.156 &  7.154.700 \\
Regular costs         & 2.578.220 & 2.578.220 &  2.578.220 \\
Bank loan (7\%)       & 2.931.696 & 2.647.925 &  2.343.641 \\
Payback               &         0 &   480.000 &    480.000 \\
Balance               &    40.530 &           &            \\  
\hline                                                      
Total                 &           &           &            \\
\end{tabular}
\caption{Summary of operation budget}
\label{opebud}
\end{table}
