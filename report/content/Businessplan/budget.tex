\subsection{Budget}
\label{budget_label}
The budget has been divided into a development budget and a operation budget. 
The development budget in table \ref{devbud} shows how much money we need for our concept to be ready for production. 
We assume that if we do not get any funding then the bank loan during the development phase will be with exempt from payments, including interest payments, until we start production. 

The main costs of the development budget is the wages, establishing, rent and some unexpected costs. 
The basis: \begin{itemize}
\item[-] for the wage is that the team (7 persons) will be working full-time (160.33 h/month) to a wage equal to the minimum salary for a third year student (172 DDK/h\cite{ida-salary}).
\item[-] for establishing is covering office fittings (37,000 DKK) and application for a patent/registration.
\item[-] for rent is a price check for renting commercial premises in Odense, Denmark. Premises with plenty of space is available at 5,000 DKK.\cite{rent_prices}
\item[-] for the interest of the bank loan of 7 \% is that the concept is with some risks \ref{risk} , if there were no risks the interest would be lower. 
\end{itemize} 
The cost are listed in table \ref{devbud}, a full description can be found in appendix \ref{detailed_budget}. 
As it appears in the table we need 2,713,824 DKK to develop our concept. 
If no funding is granted a bank loan of 2,850,000 is needed for maintaining positive bank balance.

\begin{table}[h!]
\centering
\begin{tabular}{l r}
Development budget      & (DKK)              \\
\hline                                       
Variable costs:         &                    \\
Bill of materials       &    28,500        \\
Regular costs:          &                           \\
Wage                    &    2,316,448           \\
Rent                    &    75,000          \\
Establishment           &    87,000          \\
Others                  &    79,000          \\
Unexpected costs (5\%)  &    127,877         \\
\hline                      
Total                   &    2,713,824           \\
\end{tabular}
\caption{Summary of development budget}
\label{devbud}
\end{table}


The operation budget shows the cash-flow when we start producing and selling our product. We have made a budget for the first two years after the development phase. The operation budget consist of turnover, variable costs, regular costs and the bank loan if no funding is provided. We have made the following assumptions:
\begin{itemize}
\item[-] The first year we guess around 60 units and the next year around twice as much.
\item[-] Bill of materials is proportional with units sold.
\item[-] Bank loan is needed. A 7 percent in annual interest because there is some risks.
\item[-] No payments on loan the first 6 months.
\item[-] 40,000 DKK in payback per month the second quarter of first year, followed with 80,000 DKK per month the rest of the loan period.
\end{itemize}
With the stated assumptions we will get a payback time on approximately 6 years (see appendix \ref{development_budget}).
A summery of the operation budget is shown in table \ref{opebud} (for full description see appendix \ref{development_budget}.
The budget is fragile, because with only 10 percent less unit sale during the two years of operation, will result in a negative bank balance in second and third year of -100,000 DKK and -560,000 DKK respectively. An increase of product price or lowering the payback amounts\footnote{Meaning a longer payback period} could solve this.  

\begin{table}[h!]
\centering
\begin{tabular}{l r r r r}
Operation budget      &            &              &             &    \\
                      & year 1     & year 2       & year 3      &    \\
\hline                                                               
Units sold            &          0 &        60   &         114  &    \\
Personnel             &          7 &         7   &          13  &    \\ 
\hline                                              
Turnover              &          0 & 4,800,000   &   9,120,000  & DKK\\
Variable costs        &     28,500 & 2,453,040   &   4,660,776  & DKK\\
Regular costs         &  2,685,324 & 1,798,032   &   3,180,666  & DKK\\
Bank loan (7\%)       &  3,056,026 & 3,047,532   &   2,297,787  & DKK\\
Payback               &          0 &   240,000   &     960,000  & DKK\\
Balance               &    136,092 &   213,429   &     321,576  & DKK\\  
% \hline                                                        
% Total                 &             &          &            \\
\end{tabular}
\caption{Summary of operation budget}
\label{opebud}
\end{table}

%%%%%%%%%%%%%%%%% Skal lige have noget med omkring at hvis vi sælger 10 procent mindre endheder får vi et balance underskud første og andet år på hhv 100.000 og 550.000....
