\section{Budget}
\label{budget_label}
The budget has been divided into a development budget and a operation budget. The development budget \todo{ref to full budget in appendix} shows how much money we need for our concept to be ready for production. We assume that if we don't get any funding then the bank loan during the development phase will be with exempt from payments until we start production. The main costs of the development budget is the wages, establishing, rent and some unexpected costs. 
The basis: \begin{itemize}
\item[-] for the wage is that the team (7 persons) will be working full-time (160,33 h/month) to a wage equal to the minimum salary for a third year student (172 DDK/h\footnote{http://english.ida.dk/salary/minimum-salary}).
\item[-] for establishing is covering office fittings (37000 DKK) and application for a patent/registration.
\item[-] for rent is a price check for renting commercial premises in Odense, Denmark. Premises with plenty of space is available at 5000 DKK.\footnote{http://www.lokalebasen.dk/leje/erhvervslokaler}
\item[-] for the interest of the bank loan of 7 \% is that the concept is with some risks , if there were no risks the interest would be lower. 
\end{itemize} 
The cost are listed in table \ref{devbud}, a full description can be found in appendix \todo{ref}. As it appears in the table we need 2.706.470 DKK to develop our concept. If no funding is granted a bank loan of 2.750.000 should be sufficient.

\begin{table}[h!]
\label{devbud}
\centering
\begin{tabular}{l c r}
Development budget &&(DKK)\\
\hline
Variable costs: Bill of materials	  	&	& \\
Bill of materials	  					&	& 25.500  \\
Regular costs: 						  	&  	& 		\\
Wage    								&	& 2.316.448\\
Rent 									&   & 75.000\\
Establishment							&   & 87.000\\
Others				 					&   & 79.000\\
Unexpected costs (5\%) 					&   & 123.522\\
\hline
Total         							&   & 2.706.470\\
\end{tabular}
\caption{Summary of development budget}
\end{table}

The operation budget shows the cash-flow when we start producing and selling our product. We have made a budget for the first two years after the development phase. The operation budget consist of turnover, variable costs, regular costs and the bank loan if no funding is provided. We have made the following assumptions:
\begin{itemize}
\item[-] One unit is sold for per month until month 18 where two is sold per month. This is because of the relative small marked and it will take some time to get an impact on the market.
\item[-] Bill of materials is proportional with units sold.
\item[-] Regular cost stays the same as during the development phase.
\item[-] 7 percent in interest on bank loan because there is some risks. This gives a payback time on approximately 6 years.
\end{itemize}
A summery of the operation budget is shown in table \ref{opebud}, a full description can be found in appendix \todo{ref}.
\begin{table}[h!]
\label{opebud}
\centering
\begin{tabular}{l r r r}
Operation budget						& 				&\hfill(DKK)\\
						&	year 1		& year 2		& year 3 \\
\hline
Turnover				&				& 3.600.000		& 5.100.000 \\
Variable costs	  		&				& 314.256		& 441.75 \\
Regular costs			&			  	& 2.578.220 	& 2.578.220	 \\
Bank loan 7\%			& 2.931.696 	& 2.647.925 	& 2.343.641\\

\hline
Total         							&   			&\\
\end{tabular}
\caption{Summary of operation budget}
\end{table}

\subsection{Introduction [better name needed]}
For the success of this product it is crucial that the price stays competitive. Offline programming methods currently in use today\footnote{Information courtesy of Valk Welding} costs 100.000 DKK, software and training included. Our product should stay within this budget or offer significant advantages over competition.
\subsection{Bill of Materials}
Below is a rough estimate of the budget for each component needed to produce the robot:
\begin{center}
\begin{tabular}{l c r}
Product           &   & Budget (DKK)\\
\hline
Camera            & : & 2.000  \\
Laser scanner     & : & 5.000 \\
Custom Housing    & : & 5.000\\
Embedded Hardware & : & 2.000\\
Software          & : & 5.000\\
UV flash          & : & 2.000\\
Interface         & : & 3.000\\
Marker            & : & 500\\[0.2cm]
\hline
Total             & : & 25.500\\ 
\end{tabular}
\end{center}

%\begin{table}
%\label{devbudgetshort}
%\begin{tabular}{•}
%
%\end{tabular}
%\caption{}
%\end{table}