\subsection{Sales and Marketing}
This section will explain how we intend on selling our product and which methods of distribution are going to be used.
\subsubsection{Sales and Distribution Channels}
The product is an add-on to existing welding robots. Therefore we have chosen not to sell the product directly to the end user. 
Instead the product will be sold to distributors such as Valk Welding, which sell robotic solutions to the industry. 
Their costumers will then have the possibility of choosing our product as an add-on when buying a new robot.
Valk Welding already have a costumer base, so we do not need to spend much energy creating new contacts. 
\subsubsection{Sales Activities}
As stated, we do not need to contact the users directly. 
We do, however, need to get an agreement with the company selling robot welding solutions.
These agreements will be achieved by offering distributers significant shares of the profit.
\subsubsection{Marketing Activities}
Taking advantage of the fact that Valk Welding already has a costumer base which may benefit from our product, will help in selling our product.
However, to make sure we get a grip on the market, we will use some resources to accompany Valk Welding to exhibitions, to make sure there is someone there who knows all the details about the product, and to convince costumers that the product is as good as we claim. 
\subsubsection{Core Message and Positioning}
It is important that we communicate to Valk Welding that we are adding to their existing product, so that they will want to sell our product. Our business strategy relies on having partners already settled in the industry.
The product should be sold with the promise of less human involvement in the welding process and higher flexibility, no two weldings needs to be alike.
