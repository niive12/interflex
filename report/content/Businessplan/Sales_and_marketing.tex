\subsection{Sales and Marketing}
\subsubsection{Sales and Distribution Channels}
For our product to function, it needs to be attached to a robotic arm, therefore we have chosen not to sell the product seperately to the user. 
In stead we are going to sell the product to a company like Valk Welding which sells robotic solutions to the industry. 
The idea is then, that the costumers have the possibility to choose our product when buying a new robot, and we will then get some of the profit. 
Valk Welding already have a costumer base, so we do not need to spend much energy creating new contacts. 
\subsubsection{Sales Activities}
As stated, we do not need to contact the users directly. 
We do, however, need to get an agreement with the company selling robotic solutions. 
We plan on doing that through phone conversations and by setting up meetings. 
\subsubsection{Marketing Activities}
We expect that the product will be easy to sell, since Valk Welding already has a costumer base, which chould benefit from buying our product. 
However, to make sure we get a grib in the market, we will use some resources to accompany Valk Welding to exhibitions, to make sure there is someone there who knows all the details about the product, and to convince the costumers that the product is as good as we think it is. 
\subsubsection{Core Message and Positioning}
It is important that we communicate to Valk Welding, that we are only adding on something to their product, so that they will want to sell our product, because without a company like them, we are in trouble. 
When we have convinced them, our product sould be sold by the qualities, that human will be less involved, and that it can do flexible solutions, so no two weldings needs to be alike. 
