\subsection{Kirstine}
My student to student teaching was about lasers. Due to the short time I had, it was only a general introduction, the listeners had heard of Bohr's atomic model, wavetheory and other pre-university physics. For my presentation I wanted to get the listeners as excited about lasers as I am, so I used different small demonstrations to prove my point and inprove the understanding of the subject. I have experience in teaching laser physics to high school students, so for me this was not as big of a challenge as it is to some, but it was different way that the listeneres were at a higher level than high school students, and also, we needed some of the knowledge to work with laser sensors in our project.

The main goal of this presentation was to give general introductions to:\\
-How a laser works.\\
-What it means to have different colors and how are they created.\\
-Interference.\\
-The dobbler effect.\\
-Applications of lasers.\\


I used many different methods to explain the teori in my presentation. \\
-The presentation was build on a power point show, to give structure, and to make sure I remembered everything. The power point included headlines of what I needed to talk about, pictures and figures explaining the phenomena.\\
-One of the pictures in the power point is a game, where the audience has to stare at a cross. Around the cross there is four different colored circles. When the circles disappear, the audience will see the complimentary color of the circles, even though nothing is there. It is a mind trick, but it keeps people focused. \\
-Since the presentation was about lasers, I found it natural to bring some to the presentation. I had a red, a blue, and a green one. I used them in a few experiments and to show something about divergence of the beam. \\
-One experiment I did was having flourescent liquid in a glass and water in another glass, to show the spontaneous emission. I also showed this by using the flourescent stars many kids have in their room. Using the blue laser these can be activated, but using the red and green will not work. This shows something about how important it is to use the right wavelength. Also, most people find this experiment very interesting, since they recognize the stars. 
-Powerful lasers are very dangerous, and hence, not all interesting eperiments can be done in a regular classroom. I therefore had a few videoes showing lasers blowing baloons, which is good for explaining color filtering. 


During the presentation I felt like all the listeners were very interested, and maybe wanted me to explain things more throughly. I tried to answer the questions, but there was not time enough for the through discussion of all things. 
I believe all the listernes were very happy with the presentation. There were no negative comments. 


It was difficult to direct the presentation towards the audience because they had different knowledge within the subject prior to the presentation. If I had to do this presentation again, I think, I would focus it more on the exact uses of lasers in our project. This would make it relavant to everyone not depending on their level of knowledge. Also, it would not be too difficult for them, because they would be able to relate to the subject. 
