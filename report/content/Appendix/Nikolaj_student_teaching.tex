\subsection{Nikolaj}
I am studying robotic systems and as part of my education I had a course on classical control theory.

Control theory works with creating models and controllers for linear systems.
\subsubsection{Session topics}
The quick intro touched on what control diagrams looked like and how open loop and closed loop systems behave on an intuitive level.

To show an example of where transfer functions exist in the real world, the transfer function for a motor was calculated.
We talked about the different ways to represent a transfer function.
Stability was defined and examples of unstable systems was given.
We talked about how to read a pole zero transfer function and what pole positions meant for stability and how it affects the system in the time domain.
We talked about overshoot, damping and settling time.
We talked about what second order systems look like in the time domain.
Then an example was calculated to show how you would design a controller given requirements to it's performance in the time domain.

\begin{figure}[h]
\centering
\begin{subfigure}[c]{0.4\textwidth}
\centering
  \begin{tikzpicture}[node distance = 2 cm]
  \node [input, name=input] {};
\node [sum,right of= input](sum) {};
\node [block, right of=sum] (K) {\(K\)};
\node [block, right of=K] (sys) {\(G(s)\)};
\node [output, right of= sys] (output){};

\draw[->] (input) -- node[pos=0.9, above] {+} (sum);
\draw[->] (sum) -- (K);
\draw[->] (K) -- (sys);
\draw[->] (sys) -- node[name=feedback] {} (output);

\node at (0,2) {} ;
\node at (0,-2) {} ;

\node[yshift=0.25cm] at (input) {in};
\node[yshift=0.25cm] at (output) {out};
\draw[->] (feedback.center) |- ++(-2.5,-1.5) node[block, name=H] {H};
\draw[->] (H) -| node[pos=0.9, left] {-} (sum);
  \end{tikzpicture}
  \caption{closed loop system}
\end{subfigure}\hfill
\begin{subfigure}[c]{0.4\textwidth}
\centering
  \begin{tikzpicture}
    \draw
      (0,-1) -- (0,1) node[above] {\(j\omega\)}
      (-4,0) -- (1,0) node[right] {\(\sigma\)}
    ;

    \node[pole] at (-1,0) {};
    \node[pole] at (-2,0) {} ;
    \node[pole] at (-3,0) {} ;
    
    \draw[dashed,->] (-2,0) -- (0,3.46);
    \draw[dashed,->] (-2,0) -- (0,-3.46);
    \draw (-1.5,0) arc(0:60:0.5);
    
    \draw[very thick] (-2,0) -- (-1.5,0) to[out=90, in=240] (-1,1.73) -- (0,3.46);
    \draw[very thick] (-1,0) -- (-1.5,0) to[out=-90, in=-240] (-1,-1.73) -- (0,-3.46);
    \draw[very thick,->] (-3,0) -- (-3.9,0);
  \end{tikzpicture}
  \caption{Root Locus}
\end{subfigure}
\caption{Topics introduced in ``student to student'' teaching about control theory}
\end{figure}

Root Locus was defined and it was shown how to read a Root Locus plot and how to draw it by hand. 
We talked about PID controllers and what the parameters meant. 
We talked about systems that uses PID in the real world.
We talked about ways to chose parameters for PID using the Zigler Nichols method.

\subsubsection{Evaluation}
A lot of topics were chosen and it was impossible to get explore a single topic withe the given time constraints.

This made it possible to give an overview so people with little to no prior knowledge of control theory could visualize what is possible and how to design a controller.

The used math was not explained in this session as all the students knew about this already.

This was well received and people seemed to follow the conclusions without going into detail about Laplas transforms and second order systems.

The illustrations used was either examples from the book \todo[inline]{Control theory S2S: Book name and numbers} or drawn myself.

There were a lot of questions during the presentation which shows that they were interested in the subject

The response to the session was that it was a bit more technical than they expected. 
But the pace was fine and it was easy to follow. 





a. Topics of your session. 
b. Evaluation of the session in general, the teaching materials, the relevance of the topics and planning and conduction of the session. 
c. The material used in the session might be enclosed in Appendix. 

Documentation and feedback on/evaluation of your lessons from your team
members must be included in the project report by each subgroup of
students. 