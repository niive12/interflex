\subsection{Thomas Søndergaard Christensen}
As a part of Experts in Teams we were required to teach a topic relevant to the development process of our product to the team. My main field being robotics, I decided to teach the topic of robot motion planning. My goal was to give an overview of the basic techniques used to plan robot motion, as well as some of the difficulties that you may be faced with when planning robot motion.
\subsubsection{Session Topics}
Initially a (very) brief overview of the different types of robots was given, three examples were used: Robot arms from the automotive industry, simple robot vacuum cleaners and finally, a team of football-playing robots competing for the Robocup.
Two very commonly used terms in robotics are the work-space and configuration-space of a robot. Using the example of a robot capable of motion in 2D space I explained the usefulness and limitations of the two concepts. Additionally I explained (or attempted to) the implications to the configuration-space of adding an obstacle in the work-space of a robot.
Lastly I presented the wavefront algorithm, explaining its workings and showing examples of applications used in class.
\subsubsection{Evaluation}
\paragraph{Group:}~\\
Generally people found the structure of the teaching session logical and easily understood. Additionally, the examples given were well-received and made a good job of describing the concepts. Some hinted that the difficulty could have been increased without it being an issue.
\paragraph{Own Reflections:}~\\
When deciding what material to include and what not to include I found it difficult to determine what level i would be teaching at. Given the limited time of the presentation, I did not want to attempt teaching subjects that require that some knowledge about robotic motion is already present. While my team did not necessarily have knowledge about my chosen subject in advance, they had had courses dealing with similar issues prior to my teaching session.\\
My experience from earlier student-to-student teachings told me that there would be many questions during the session, this led me to choose to  limit the length of my presentation to roughly 30 minutes, allowing time for questions. However, not a single question was asked during the presentation, meaning that my session fell short of the allowed 45 minutes. In hindsight I should have included an extra topic in case of the event that no questions were asked.\\
In general I am satisfied with the outcome of my teaching session. I believe the information was relevant to the project at hand and it was well received by my team.
