\section{Valk Welding Meeting Summary}\label{Valk_meeting_resume}
We scheduled a meeting with Valk Welding Wednesday the 12th of november.

We talked to Marcel Dingemanse, branch manager of this company in Denmark.

Valk Welding work with Panasonic robots and sell complete solutions that a welding production can set up and work with. 
The robots are currently programmed either by using an online or an offline method. 
 
Even though each system is different, the yearly budget could be approximated to 1 million per robot they sold.
Such a system can produce the same amount of units as 5 welders, but their jobs is replaced by a handler and a programmer. 
Valk Welding also sell a training program that costs 100,000 DKK.
This training course takes 5 days where they work with the online programming and 3 days where they work with the offline programming.

The online method consists of physically moving the robot into position. Then a button is pushed so the robot knows it has to visit this point.
Then it is moved to the next position until the entire movement is recorded.
When this is completed the robot will move through all these points and the programming is complete.
This is very intuitive and you make sure that the path is possible.
The downside to this is that the robot cannot work while it is being programmed.

The offline method depends on a 3D CAD model of the item. 
By clicking on the CAD drawing the programmer gets coordinates and can use this to develop the welding program.
Valk Welding is working with customers and Panasonic to make macros that help the programmers make the repetitive algorithms easier.
We saw a demonstration video of the process of making a program that welds a girder. 
A girder only has a few welds and all these welds are straight lines. 
Generating the program for a new girder took less than 15 minutes. 
All these improvements make this process a very fast option and since it does not require moving the robot this can be done while the robot is working on other tasks.
This is the preferred way to program a welding robot in Denmark but there is still a stigma about using this method in other countries.

Both methods were used in Sweden and they were not that willing to change as the Danish welders.

There is not much to improve on the precision of the robot.
In reality is the quality of the weld irrelevant. 
An experienced welder might do a better job than the robot but the robot is precise enough for the welding market with current technology.
A current challenge the welding robot is facing is a good way to detect the placement of certain features as the CAD model is not 100\% accurate.
The way they currently solve this is by slowly moving the robot closer towards the feature and then detecting the metal by touching with the welding torch. 
This is done to both sides and the corner is calculated from this data.
The robot has to change its welding wire after each detection so this makes the weld take a longer time.
The offline welding is 5-10 times faster to program than using the online programming method. 

Valk Welding has departments all over the world. 
Marcel Dingemanse did not have any information on how much the other departments sold but last year they sold 22 welding robots in Denmark and 100 in Sweden.
The main thing customers look for is a fast production.
They want to be able to deliver as much as possible and the robots need to be fast to do this.
The robot is expensive but if it works all the time it is a lot cheaper than human workers.
So the customers do not want a long downtime when they switch to a new type of welding.

All this information was used in the project, it helped to understand how the company works, in what they focus when they develop the technology and which are the main customers.
