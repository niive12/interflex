\subsection{Xabier Martinez Beneitez}
To tell the truth, I wrote another report, but seeing the presentations of my colleagues I thought that I should change it.
\subsubsection{Session topics and used methodology}
My presentation covered the basics of static systems. I thought it would be interesting to talk about this, because it is essential to know, in order to understand the behaviour of most of the things, and even if you work in the field of robotics, as it is our case.\\\\
I tried to use a Power Point to teach all the information I wanted to show. I give some brief definitions about basic concepts, in order to understand better how to solve the exercises. I thought it was going to be a good idea to do this in a more dynamic way, where the rest of the people had the opportunity to express their doubts about these problems. It did not work too bad, people were interested in the problems and I tried to solve all the doubts they had. I tried to use a linear development to present my ideas, where ever they were incorporating new knowledge. It is for this reason that they did not find it very difficult to follow the steps that I was giving, because as they also said, the structure of my exhibition was one of the presentation's strengths. Still, it must be said that some of them already had knowledge of these issues, so I help in some cases to remember it.\\\\
Every time I explained a new concept, I tried to pair it with a problem, so they could see the practical side of the theory presented. One thing that was attributed to the resolution method was that it was not performed the complete resolution of the problems, i.e. I focused more on trying to explain the meaning of the equations that the mere mathematical development. I think that was right, many of them thought the same as the time was limited and the mathematical resolution takes time. But it is true that maybe I should have done at least one complete problem. I also tried to use lots of images and graphs, so it made more visual and easier to understand.\\\\
\subsubsection{Evaluation and personal opinion}
This was all about the exposed material and the way I tried to expose it. But what about how I saw myself doing the presentation? I felt a little nervous when I started with the presentation, because in Spain, the teaching method has a higher theoretical component, so we are not used to do this kind of presentations. But as time went on, I was feeling better, more comfortable and I could explain things in a more calm and clear way. I was quite happy when I saw that they were asking lots of question, because that means that they were interested in what I was talking about.\\\\
But one of the most important things I wanted to comment on this report is all I have learned from this 'Student to Student Teaching'. As I said at the beginning of this text, I rewrote this report, as my vision about how well I did changed when I saw some of my classmates presentations. In my opinion, I did an standard presentation without too much innovation and risking, that is because I had not seen myself very qualified to do so, having no practice in making almost presentations. I knew that I should improve my English level, in order to express myself in a better way. Seeing the presentations of my classmates I learned a lot. I was amazed about how much some people contacted with the listeners while they were doing the presentation, how they raised and how using daily examples were able to present a subject that was not a really exiting one, in a really attractive way.\\\\
They told many stories, many examples and in some cases, they brought some real examples to our room. Seeing all this, I realized how I could improve my presentations, I have too much to learn, but also it motivated me because, I have my faults clearer and I know how I can improve. Improving English will make easier for me to express my ideas, which will help to engage people. I have to focus more on presenting less theoretical and more practical content.\\\\
In conclusion, I realized I made failures thanks to feedback from my peers. I also understood all the things I had done well, but also I learned a lot from the presentations of my classmates. For all this, I think it was a positive activity, which I could benefit greatly.
